%%%%%%%%%%%%%%%%%%%%%%%%%%%%%%%%%%%%%%%%%
% Short Sectioned Assignment
% LaTeX Template
% Version 1.0 (5/5/12)
%
% This template has been downloaded from:
% http://www.LaTeXTemplates.com
%
% Original author:
% Frits Wenneker (http://www.howtotex.com)
%
% License:
% CC BY-NC-SA 3.0 (http://creativecommons.org/licenses/by-nc-sa/3.0/)
%
%%%%%%%%%%%%%%%%%%%%%%%%%%%%%%%%%%%%%%%%%

%----------------------------------------------------------------------------------------
%	PACKAGES AND OTHER DOCUMENT CONFIGURATIONS
%----------------------------------------------------------------------------------------

\documentclass[paper=a4, fontsize=11pt]{scrartcl} % A4 paper and 11pt font size

\usepackage[T1]{fontenc} % Use 8-bit encoding that has 256 glyphs
\usepackage{fourier} % Use the Adobe Utopia font for the document - comment this line to return to the LaTeX default
\usepackage[english]{babel} % English language/hyphenation
\usepackage{amsmath,amsfonts,amsthm} % Math packages

\usepackage{lipsum} % Used for inserting dummy 'Lorem ipsum' text into the template

\usepackage{enumitem}
\usepackage{mathtools}
\DeclarePairedDelimiter\ceil{\lceil}{\rceil}
\DeclarePairedDelimiter\floor{\lfloor}{\rfloor}

\usepackage{sectsty} % Allows customizing section commands
\allsectionsfont{\centering \normalfont\scshape} % Make all sections centered, the default font and small caps

\usepackage{fancyhdr} % Custom headers and footers
\pagestyle{fancyplain} % Makes all pages in the document conform to the custom headers and footers
\fancyhead{} % No page header - if you want one, create it in the same way as the footers below
\fancyfoot[L]{} % Empty left footer
\fancyfoot[C]{} % Empty center footer
\fancyfoot[R]{\thepage} % Page numbering for right footer
\renewcommand{\headrulewidth}{0pt} % Remove header underlines
\renewcommand{\footrulewidth}{0pt} % Remove footer underlines
\setlength{\headheight}{13.6pt} % Customize the height of the header

\numberwithin{equation}{section} % Number equations within sections (i.e. 1.1, 1.2, 2.1, 2.2 instead of 1, 2, 3, 4)
\numberwithin{figure}{section} % Number figures within sections (i.e. 1.1, 1.2, 2.1, 2.2 instead of 1, 2, 3, 4)
\numberwithin{table}{section} % Number tables within sections (i.e. 1.1, 1.2, 2.1, 2.2 instead of 1, 2, 3, 4)

\setlength\parindent{0pt} % Removes all indentation from paragraphs - comment this line for an assignment with lots of text

%----------------------------------------------------------------------------------------
%	TITLE SECTION
%----------------------------------------------------------------------------------------

\newcommand{\horrule}[1]{\rule{\linewidth}{#1}} % Create horizontal rule command with 1 argument of height

\title{	
\normalfont \normalsize 
\textsc{University of Toronto, Department of ECE} \\ [25pt] % Your university, school and/or department name(s)
\horrule{0.5pt} \\[0.4cm] % Thin top horizontal rule
\huge ECE1762 - Homework 1 \\ % The assignment title
\horrule{2pt} \\[0.5cm] % Thick bottom horizontal rule
}

\author{Xinyun Lv, Yang Wang} % Your name

\date{\normalsize\today} % Today's date or a custom date

\begin{document}

\maketitle % Print the title

%----------------------------------------------------------------------------------------
%	PROBLEM 1
%----------------------------------------------------------------------------------------

\section{Problem I}
Sort the following functions from asymptotically smallest to asymptotically largest.

\begin{center}
	$ 2^{log_{10}{n}} \quad log_{lg{n}}{n} \quad  lg(nlog{n})  \quad  n \quad (\sqrt{2})^{lg{n}} \quad (lglgn)^{lglgn} $
\end{center}

%----------------------------------------------------------------------------------------
%	PROBLEM 2
%----------------------------------------------------------------------------------------
%----------------------------------------------------------------------------------------
\section{Problem II}
Show that any sequence of $n^3 + 1$ numbers contains either
\begin{itemize}
	\item a strictly-increasing subsequence of length $n + 1$.
	\item a strictly-decreasing subsequence of length $n + 1$, or
	\item $n + 1$ elements with the same value.
\end{itemize}

%----------------------------------------------------------------------------------------
%	PROBLEM 3
%----------------------------------------------------------------------------------------
%----------------------------------------------------------------------------------------
\section{Problem III}
 Solve the following recurrences. State tight asymptotic bounds for each function in the form $\Theta(f(n))$ for some recognizable function $f(n)$. Prove your answer. Assume reasonable but nontrivial base cases if none are supplied.

\begin{enumerate}[label=(\alph*)]
	\item $A(n)=2A(n/4)+nloglogn$
	\item $B(n) = B(n/2) + log n$
	\item $C(n)=3C(n/2)+nlogn$
	\item $F(n)=F(\floor*{logn})+logn$
\end{enumerate}

%----------------------------------------------------------------------------------------
%	PROBLEM 4
%----------------------------------------------------------------------------------------
%----------------------------------------------------------------------------------------
\section{Problem IV}
$m$ balls are thrown into $n$ bins (independently) so that each ball is equally likely to fall into any of the bins. Estimate as precisely as you can the smallest number $m$ (as a function of $n$) so that the probability of all balls falling into different bins is smaller than $\frac{1}{n^{c}}$, for a fixed constant $c>0$.

%----------------------------------------------------------------------------------------
%	PROBLEM 5
%----------------------------------------------------------------------------------------
%----------------------------------------------------------------------------------------
\section{Problem V}
Give a combinatorial argument to prove that
\begin{center}

\end{center}









\end{document}